
\documentclass[parskip=full,11pt]{scrartcl}
\usepackage[utf8]{inputenc}

\title{Performance Dashboard for Continuous Benchmarking of HPC Libraries}
\author{Chingun Ariunbat, Maximilian Schik, Walter Alexander B\"ottcher,\\ Darius Schefer, Jamil Bagga}

% section numbers in margins:
\renewcommand\sectionlinesformat[4]{\makebox[0pt][r]{#3}#4}

\newcommand{\scenario}[2]{\textbf{Scenario name:} #1 \\
\textbf{Participating actor:} #2}

% header & footer
\usepackage{scrlayer-scrpage}
\lofoot{\today}
\refoot{\today}
\pagestyle{scrheadings}

\usepackage[sfdefault,light]{roboto}
\usepackage[T1]{fontenc}
\usepackage[english]{babel}
\usepackage[yyyymmdd]{datetime} % must be after babel
\renewcommand{\dateseparator}{-} % ISO8601 date format
\usepackage[colorlinks=true, linkcolor=blue]{hyperref}
\usepackage{amsmath} % for $\text{}$
\usepackage[nameinlink]{cleveref}
\crefname{figure}{Abb}{Abb}
\usepackage[section]{placeins}
\usepackage{xcolor}
\usepackage[nonumberlist]{glossaries}     % provides glossary commands
\usepackage{graphicx}
\hypersetup{
	pdftitle={Pflichtenheft},
	bookmarks=true,
}
\usepackage{csquotes}

\makenoidxglossaries

\makenoidxglossaries

\newglossaryentry{user}
{
	name=user,
	plural=users,
	description={Person working on the project that gets benchmarked}
}

\newglossaryentry{configuration}
{
	name=configuration,
	plural=configurations,
	description={A complete description of a \gls{plot}. It contains all the necessary information except the benchmark data}
}

\newglossaryentry{template}
{
	name=template,
	plural=templates,
	description={A partial configuration of a \gls{plot}, It contains preconfigured values, but leaves others blank for the user to costumize}
}

\newglossaryentry{plot}
{
	name=plot,
	plural=plots,
	description={A graphical representation of benchmark data (e.g. graphs)}
}

\newglossaryentry{benchmarking system}
{
	name=benchmarking system,
	plural=benchmarking systems,
	description={System that runs benchmark for a code base}
}

\newglossaryentry{REST API}
{
	name=REST API,
	plural=REST APIs,
	description={Representational state transfer API - Software architectural style used to implement and provide Web Services after certain guidelines}
}

\newacronym{ci}{CI}{Continuous Integration}

\newacronym{json}{JSON}{JavaScript Object Notation}

\newcommand\urlpart[2]{$\underbrace{\text{\texttt{#1}}}_{\text{#2}}$}

\usepackage{pflichtenheft}

\begin{document}
\maketitle

\section{Introduction}
Glossary acronym example: \\
\acrshort{ci} \\
\acrlong{ci} \\
\acrfull{ci}

\begin{center}
\urlpart{http}{protocol}%
\texttt{://}%
\urlpart{web.io}{host}%
\texttt{/}%
\urlpart{index}{path}%
\texttt{?}%
\urlpart{argument=somevalue}{parameter}%
\texttt{\#}%
\urlpart{theAnchor}{fragment}
\end{center}

more placeholder

\section{Goals}
% Diese Section sollte kurz und knapp "fuer Manager" sein
% und auf eine Seite passen.

\subsection{Required}

\criterium{heading}{crt:length}

yet more placeholder

\criterium{Schnelle Weiterleitung Kurz- zu Lang-URL}{crt:fast}

\criterium{Authentifizieren mit E-Mail oder Facebook}{crt:login}

\criterium{Rechtlichte Vorgaben werden eingehalten}{crt:tmg}

template

\subsection{Optional}

\criteriumOptional{Authentifizieren mit Github}{crt:github}

\criteriumOptional{Seite mit Betreiberinfo}{crt:about}

template

\subsection{Limitation}

\criteriumNot{Keine Wahl Kurz-URL}{crt:no-choice}

template

\pagebreak
%%%%%%%%%%%%%%
\section{Usage}

template
\section{Product Environment}

template

\section{Requirements}

\subsection{Functional Requirements}

\functionality{Generation of visualizations from a single datapoint}{fnc:generate-single}
\fulfills{crt:vis}

\functionality{Generation of visualizations from multiple datapoints}{fnc:generate-multiple}
\fulfills{crt:comp}

\functionality{Configuration of visualization}{fnc:configure}
\fulfills{crt:vis}
\fulfills{crt:comp}

\functionality{Display visualizations}{fnc:display}
\fulfills{crt:vis}
\fulfills{crt:comp}

\functionality{Storage of benchmark results}{fnc:storage}
\fulfills{crt:storage}

\functionality{Calculation of performance metrics}{fnc:calc}
\fulfills{crt:overview}

\functionality{Annotation of commit history with performance metrics}{fnc:overview}
\fulfills{crt:overview}

\functionality{Robustness against errors}{fnc:error}

\functionality{Storage of templates in cookies}{fnc:store-template}
\fulfills{crt:opt-templates}

\functionality{Creation of templates in cookies}{fnc:create-template}
\fulfills{crt:opt-templates}

\functionality{Usage of templates}{fnc:use-template}
\fulfills{crt:opt-templates}

\functionality{Creation of a shareable link}{fnc:share-link}
\fulfills{crt:opt-export-share}

\functionality{Creation of a downloadable file from a visualization}{fnc:export}
\fulfills{crt:opt-export-share}

\functionality{Tracking performance}{fnc:track}
\fulfills{crt:opt-track}

\functionality{Notifying external webservices}{fnc:notify}
\fulfills{crt:opt-track}

\subsection{Nonfunctional Requirements}

\nonFunctionality{Available configuration options are understandable without previous knowledge}{nfc:intuitiv}

\nonFunctionality{Storage of benchmark results is persistent}{nfc:persistent}

\nonFunctionality{Generation of visualizations does not take more than 15 seconds}{nfc:runtime}

\nonFunctionality{Extension with new visualization types is simple}{nfc:extension}

\subsection{Product Data}

\productData{Benchmark Results (Name in progress)}{pdt:benchmark-results}
Format: JSON/CSV \\
Description: 
\begin{itemize}
	\item saved on server
	\item algorithm result data (time, storage, accuracy, convergence(?))
\end{itemize}

\productData{Git Histories}{pdt:git-histories}
Format: ??? (WIP) \\

\productData{\Glspl{template}}{pdt:template}
Format: JSON (?)


\section{Test Scenarios}

The test scenarios are taken from the specification (T1, ...).

\subsection{(T1) Create Plot}
Tests the basic navigation of the application and the generation of a plot from a set of benchmark configurations

\begin{center}
  \begin{tabular}{|l|ll|}
    \hline
    Create Plot & \PASS & Tested via \textcolor{orange}{Selenium} \\
    \hline
    \textbf{T1.1} & & \\
    \PASS & \textbf{State} & User has the application open and no commits selected \\
          & & The database contains benchmark results \\[.5\normalbaselineskip]
    & \textbf{Action} & User selects two commits \\[.5\normalbaselineskip]
    & \textbf{Reaction} & Commits get added to the list of selected commits \\[.5\normalbaselineskip]
    \hline
    \textbf{T1.2} & & \\
    \PASS &  \textbf{State} & User has two commits selected \\[.5\normalbaselineskip]
    & \textbf{Action} & User selects the \enquote{Create New Plot} option \\[.5\normalbaselineskip]
    & \textbf{Reaction} & The configuration panel appears \\[.5\normalbaselineskip]
    \hline
    \textbf{T1.3} & & \\
    \PASS & \textbf{State} & User has two commits selected and the configuration panel open \\[.5\normalbaselineskip]
    & \textbf{Action} & User adjusts the configuration \\[.5\normalbaselineskip]
    & \textbf{Reaction} & Configuration gets updated on screen \\[.5\normalbaselineskip]
    \hline
    \textbf{T1.4} & & \\
    \PASS & \textbf{State} & User has two commits selected and the \\
          & & configuration panel open and has it configured \\[.5\normalbaselineskip]
    & \textbf{Action} & User selects the \enquote{Generate Plot option} \\[.5\normalbaselineskip]
    & \textbf{Reaction} & The plot matching the configuration gets displayed to the user \\[.5\normalbaselineskip]
    \hline
  \end{tabular}
  \end{center}

  \subsection{(T2) Post Benchmark Results}
  Tests sending benchmark results to the system and storing it

  \begin{center}
    \begin{tabular}{ | l | l l | }
      \hline
      Post Benchmark & \PASS & Tested via \textcolor{orange}{Selenium} \\
      Results & & \\
      \hline
      \textbf{T2.1} & & \\
      \PASS &  \textbf{State} & Backend is running \\[.5\normalbaselineskip]
    & \textbf{Action} & The system receives a POST request with valid data \\[.5\normalbaselineskip]
    & \textbf{Reaction} & The data gets stored in the database \\[.5\normalbaselineskip]
    \hline
  \end{tabular}
  \end{center}

  \subsection{(T3) Error Handling}
  Tests the system’s ability to deal with malformed input data and invalid requests

  \begin{center}
    \begin{tabular}{|l|ll|}
      \hline
      Error Handling & \PASS & Tested via \textcolor{orange}{Python requests} \\
      \hline
      \textbf{T3.1} & & \\
      \PASS & \textbf{State} & Database is empty \\[.5\normalbaselineskip]
    & \textbf{Action} & System receives a POST request with malformed data for commit A \\[.5\normalbaselineskip]
    & \textbf{Reaction} & System does not store the data \\[.5\normalbaselineskip]
    \hline
    \textbf{T3.2} & & \\
    \PASS &  \textbf{State} & Database is empty \\[.5\normalbaselineskip]
    & \textbf{Action} & The system requests data for commit A \\[.5\normalbaselineskip]
    & \textbf{Reaction} & An error message gets displayed in the application \\[.5\normalbaselineskip]
    \hline
  \end{tabular}
  \end{center}
  \clearpage

  \subsection{(T4) Templates}
  Tests the creation, storage and usage of templates

  \begin{center}
    \begin{tabular}{|l|ll|}
      \hline
      Templates & \PASS & Tested via \textcolor{orange}{Selenium} \\
      \hline
      \textbf{T4.1} & & \\
      \PASS & \textbf{State} & User has commits selected and the application with the configuration \\
            & & panel open. Configuration has been changed from the default \\[.5\normalbaselineskip]
    & \textbf{Action} & User selects the \enquote{Save Template} option \\[.5\normalbaselineskip]
    & \textbf{Reaction} & Template gets downloaded \\[.5\normalbaselineskip]
    \hline
    \textbf{T4.2} & & \\
    \PASS & \textbf{State} & Template is downloaded \\[.5\normalbaselineskip]
    & \textbf{Action} & User selects the \enquote{Load Template} option \\[.5\normalbaselineskip]
    & \textbf{Reaction} & Template gets applied to the current configuration \\[.5\normalbaselineskip]
    \hline
  \end{tabular}
  \end{center}

  \subsection{(T5) Share}
  Tests the ability to share links to plot views

  \begin{center}
    \begin{tabular}{ | l | l l | }
      \hline
      Share & \PASS & Tested via \textcolor{orange}{Selenium} \\
      \hline
      \textbf{T5.1} & & \\
      \PASS &  \textbf{State} & User has the application with a plot open \\[.5\normalbaselineskip]
    & \textbf{Action} & User selects the \enquote{Share} option \\[.5\normalbaselineskip]
    & \textbf{Reaction} & A link that leads to an identical plot is copied to the clipboard \\[.5\normalbaselineskip]
    \hline
  \end{tabular}
  \end{center}
  \clearpage

  \subsection{(T6) Export}
  Tests the ability to export plots to \texttt{.png} files

  \begin{center}
    \begin{tabular}{ | l | l l | }
      \hline
      Export & \PASS & Tested via \textcolor{orange}{Selenium} \\
      \hline
      \textbf{T6.1} & & \\
      \PASS &  \textbf{State} & User has the application with a plot open \\[.5\normalbaselineskip]
    & \textbf{Action} & User selects the \enquote{Download Plot} option \\[.5\normalbaselineskip]
    & \textbf{Reaction} & The user gets prompted with a download of the plot as a \texttt{.png} image \\[.5\normalbaselineskip]
    \hline
  \end{tabular}
  \end{center}

  \subsection{(T7) Invalid Authentication}
  Authentication is not implemented (yet?). See open issues.

  \subsection{(T8) Performance Tracking}
  Tests the system's ability to detect performance drops and notify external webservices


  \begin{center}
    \begin{tabular}{ | l | l l | }
      \hline
      Performance & \PASS & Tested \textcolor{orange}{manually} \\
      Tracking & & \\
      \hline
      \textbf{T8.1} & & \\
      \PASS &  \textbf{State} & The database contains a commit with high performance scores and \\
            & & a hook for an external webservice is set up \\[.5\normalbaselineskip]
    & \textbf{Action} & System gets a POST request with data that has low performance scores \\[.5\normalbaselineskip]
    & \textbf{Reaction} & The external webserivce gets notified of the performance drop \\[.5\normalbaselineskip]
    \hline
  \end{tabular}
  \end{center}


\appendix


\printnoidxglossaries

\end{document}
